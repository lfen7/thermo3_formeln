%% Include this document with the \input command

\documentclass[a4paper, 11pt]{article}
\usepackage[a4paper, landscape, margin=1cm, twocolumn]{geometry}

% -------- Packages --------
\usepackage[utf8]{inputenc} 		% input encoding, umlauts etc.
\usepackage[T1]{fontenc}
\usepackage[ngerman]{babel} 	% german word separation

\usepackage{tikz}				% graphics
\usepackage{graphicx}		% boxes around Eqns, Formulae, etc
\usepackage{tabularx}

\usepackage{fancyhdr}		% headers
\usepackage{lscape}			% make it possible to rotate text 180deg
\usepackage{multicol}			% multiple columns
\usepackage{hyperref}			% hyperlinks
\hypersetup{
     		% bookmarks = true,
     		colorlinks = true,       				% false: boxed links; true: colored links
    		linkcolor = black,          				% color of internal links
    		citecolor = black,        				% color of links to bibliography
    		filecolor = black,      				% color of file links
    		urlcolor = black          				% color of external links
}
\usepackage{wrapfig}			% wrapping text
\usepackage{lineno}			% possibility do display line numbers
\usepackage{color}
%\usepackage{showkeys}		% UNCOMMENT BEFORE FINAL COMPILE, shows label names

\usepackage{amsmath}
\usepackage{amssymb}
\usepackage{amsfonts}		% math
\usepackage{natbib}			% citation (produed by Overleaf)

% -------- Commands --------
\newcommand{\latex}{\LaTeX \space}

% -------- Title --------
\author{Leo Fent}
\title{Title}



\begin{document}

\section*{Velocity Triangles}
\begin{itemize}
	\item You need to draw velocity triangles over a turbine rotor and compressor rotor.
	\item Be careful with signs and indices of stator and rotor.
\end{itemize}

\subsection*{Nomenclature} 
\begin{tabular}{ m{5cm} l}
	Stator exit angle & $\alpha_1$\\
	Rotor inlet angle & $\beta_1$ \\
	Rotor outlet angle & $\beta_2$ \\
	Rel. rotor inlet velocity & $W_1$ \\
	Rel. rotor outlet velocity & $W_2$ \\
   \textit{ Absolute frame of reference:} & \\
    Velocity & $C$ \\
    Angle & $\alpha$ \\
    \textit{Relative frame of reference:} & \\
    Velocity & $W$ \\
    Angle & $\beta$ \\
\end{tabular} \\

\subsection*{Calculations} 
\begin{tabular}{ m{5cm} l}
	Rotational speed & $U = \omega \cdot r = 2 \pi \cdot f \cdot r$ \\
	Mass flow & $\dot{m} = \rho \cdot A \cdot C_x $ \\
	Geometry & $C_{\theta 1 } = C_x \tan( \alpha_1 ) $ \\
		& $C_{\theta 1} = r \omega - C_x \tan \left( \beta_2 \right) $ \\
		& $\Delta h_T = U \cdot \Delta C_\theta = U \left(C_{\theta 2} - C _{\theta 1} \right) $ \\
	Euler Work & $\Delta W = \frac{\text{Power}}{\dot{m}} = \Delta h_T = \dot{m} \cdot U \cdot \Delta C_\theta$
\end{tabular}


\section*{Thermodynamics}
\begin{tabular}{ m{5cm} l}
    1st Law of Thermodynamics & $du = dq - dw$  \\
     & $\dot{W} - \dot{Q} = \dot{m} \left( h_1 - h_2 + \frac{v_1^2 - v_2^2}{\rho} + g \left(z_1 - z_2 \right) \right)$ \\
    TDS Equation 1 & $Tds = du + pdv$ \\
    TDS Equation 2 & $Tds = dh - vdp$ \\
    Enthalpy & $h = u + pv$ \\
     & $d h  = du + p dv + v dp$ \\
\end{tabular}

\section*{Ideal Gas}
\begin{tabular}{ m{5cm} l}
    Ideal Gas Equation & $pv = RT$ \\
     & $p = \rho R T$ \\
    Constant Volume & $c_v = (\partial u / \partial T)_v$ \\
     & $du = c_v (T) dT$ \\
     Constant Pressure & $c_p = (\partial h / \partial T)_p$ \\
     & $dh = c_p (T) dT$ \\
     Thermal coefficients & $R = c_p - c_v$ \\
      & $\gamma = k = \frac{c_p}{c_v}$ \\
      Specific Enthalpy of Gases & $h = u + pv = u(T) + RT = h(T)$ \\
       & $\Rightarrow$ does not depend on pressure
\end{tabular}

\subsection*{Ideal Gas – Isentropic Conditions}
\begin{tabular}{ m{5cm} l}
    $\frac{T_2}{T_1} = \left( \frac{p_2}{p_1} \right)^{ \frac{\gamma - 1}{\gamma}} $ & $\frac{T_2}{T_1} = \left( \frac{v_1}{v_2} \right)^{ \gamma -1 } $ \\
    $\frac{p_2}{p_1} = \left( \frac{v_1}{v_2} \right)^{ \gamma } $ & \\
\end{tabular}

\section*{Turbines, Compressors, Refrigerator and Heat Pump}
    \begin{tabular}{ m{5cm} l}
    Isentr. turbine efficiency & $\eta_{\text{turbine}} = \frac{h_{in} - h_{out}}{h_{in} - h_{out,s}} = \frac{\text{real work}}{\text{ideal work}} $ \\
    Isentr. compressor efficiency & $\eta_{\text{compressor}} = \frac{h_{out,s} - h_{in}}{h_{out} - h_{in}} = \frac{\text{ideal work}}{\text{real work}} $  \\
    Thermal efficiency & $\eta_{\text{thermal}} = \frac{W_{\text{out, net}}}{Q_{\text{in}}} $ \\
    Ideal pump work & $\Delta h = v \cdot \Delta p$ \\
    & \textbf{Careful:} \textit{p} in Pa, not bar! \\
    Heat pump "Leistungsziffer" & $\varepsilon_{HP} = \frac{\Dot{Q}_{out}}{W_{komp}} \left( = \frac{h_2 - h_3}{h_2 - h_1} \right) $ \\
    Refrigerator "Leistungsziffer" & $\varepsilon_{KM} = \frac{\Dot{Q}_{in}}{W_{komp}} \left( = \frac{h_1 - h_4}{h_2 - h_1} \right)$ \\
    \end{tabular}
    
\section*{General}
You should know:
\begin{itemize}
    \item The t-s diagrams by heart and be able to draw it.
    \item How to draw the p-v diagrams of the Otto and Diesel Cycle.
    \item How the Brayton Cycle works. 
    \item How the heat pump and refrigerator cycles work.
    \item Ottoprozess: Gleichraumprozess (isochor), Diesel: Gleichdruckprozess (isobar).
    
    
\end{itemize}

\end{document}
